\begin{abstract}
Embedded computing systems are being deployed on an increasingly larger scale, and it is expected that this trend will continue in the forseable future. It follows that there is a great market for developing software for embedded computing systems. In contrast to general-purpose computing, embedded computing often places strong constraints on energy consumption, performance, and cost. In order to implement these qualities in a system, software developers must make extensive use of low-level hardware functionality, and specific knowledge of the hardware components and technology used is required. 

The course TDT4258 Energy Efficient Computer Systems gives an introduction to microcontroller programming on a practical and theoretical level, with a strong focus on energy-efficiency. Through lectures and three comprehensive exercises the students get theoretical knowledge as well as hands-on experience. The development platform used is the EFM32GG-DK3750 development kit from Silicon Labs, and the tools used include energyAware Commander and the GNU Compiler Collection. In the first exercise, the students connect the prototyping board to a specially built gamepad with LEDs and buttons, and write a small program that lets the user to control the gamepad LEDs through the buttons. 

During the exercise we learned to connect the prototyping board to development software, load binary programs, and monitor program energy consumption. On the development end we learned about the GNU Compiler Collection, how to debug running code using the GNU Debugger with GDBServer, and how to program the Cortex-M3 microprocessor using the Thumb-2 instruction set. The exercise program was implemented in several steps, first using a simple polling loop, then a more sophisticated technique with interrupts, and finally with a simple countdown clock. Through this process we observed a great improvement in energy-efficiency.
\end{abstract}
