\chapter{Methodology}
% This chapter should discuss the details of your implementation for the
% assignment. Everything related to \emph{how} things were done should go here.
% Remember to avoid going into too much details, summarize appropriately and
% try to use figures/charts. Make sure you refer to the figures (such as Figure
% \ref{fig:universe}) and charts you add in the text. Avoid putting lots of
% source code here -- small code snippets are fine if you want to discuss
% something specific.

\section{Development Setup}
This section briefly describes technical details regarding the system setup we used on our development computers.

\subsection{PTXdist}
The build process of this exercise was quite involved, and it was necessary to use the PTXdist build management tool, described in section \ref{sec:ptxdist}. In order to install PTXdist on our computers, two main options were provided along with the exercise:
\begin{itemize}
  \item A virtual machine with PTXdist and cross toolchain installed.
  \item A README file with instructions on how to build and install PTXdist and cross toolchain from sources.
\end{itemize}

Initially we used a computer with limited processing resources for development, and building PTXdist from sources was therefore the favoured alternative. By following the instructions, PTXdist and cross toolchain was installed successfully in a couple of hours.

A possible third option was to use the lab computers with PTXdist and cross toolchain already installed. This option was regarded as a fall-back solution in case of technical difficulties.

\subsection{Serial Terminal Emulator}
In order to access the running operating system, we connected a laptop running a serial terminal emulator to the serial port of the EFM32GG. The program used for terminal emulation was \emph{minicom}, an open source communication program available for most Unix and Unix-like operating systems.\cite{minicom-man-page} The compendium instructions was followed to set up parameters for the serial port:
\begin{itemize}
  \item Baud rate: 115200 bps
  \item Data bits: 8
  \item 1 stop bit
  \item No parity
  \item No flow control
\end{itemize}
Connecting to the operating system via the serial port gave us access to a shell where we could type commands directly.
% TODO(maybe): Add an image of the shell

\subsection{Git Version Control}
The Git version control tool was used to manage different code revisions. The free hosting service GitHub was used to host a Git repository at \url{https://github.com/jonatanl/TDT4258}. The branch ``master'' was used for code development.



\section{Testing}
% Add content in this section that describes how you tested and verified the
% correctness of your implementation, with respect to the requirements of the
% assignment.
