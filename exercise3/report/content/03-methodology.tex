\chapter{Methodology}
% This chapter should discuss the details of your implementation for the
% assignment. Everything related to \emph{how} things were done should go here.
% Remember to avoid going into too much details, summarize appropriately and
% try to use figures/charts. Make sure you refer to the figures (such as Figure
% \ref{fig:universe}) and charts you add in the text. Avoid putting lots of
% source code here -- small code snippets are fine if you want to discuss
% something specific.

\section{System Setup}
% TODO: Write a section about system setup:
%  - development (host) platform used and target platform
%  - how ptxdist was used, configured etc.

\subsection{Serial Terminal Emulator}
In order to access the running operating system, we connected a laptop running a serial terminal emulator to the serial port of the EFM32GG. The program used for terminal emulation was \emph{minicom}, an open source communication program available for most Unix and Unix-like operating systems.\cite{minicom-man-page} The compendium instructions was followed to set up parameters for the serial port:
\begin{itemize}
  \item Baud rate: 115200 bps
  \item Data bits: 8
  \item 1 stop bit
  \item No parity
  \item No flow control
\end{itemize}
Connecting to the operating system via the serial port gave us access to a shell where we could type commands directly.
% TODO(maybe): Add an image of the shell


\section{Testing}
% Add content in this section that describes how you tested and verified the
% correctness of your implementation, with respect to the requirements of the
% assignment.
