\chapter{Background and Theory}
% This chapter should describe the theoretical background needed to understand
% and solve the problem. For instance, a description of the hardware platform
% or specific components involved in this assignment, definition of concepts
% that are important to understand the solution should be summarized here. Add
% citations to show sources whenever appropriate, LaTeX and bibliography
% managers make this easy.

% TODO: Write an introduction to ptxdist usage


\section{Operating Systems}
An operating system is a special piece of software that provides two important functions in a computer:
\begin{itemize}
  \item Managing the hardware resources.
  \item Providing a useful hardware abstraction layer for application programmers.
\end{itemize}
The core of the operating system is called the \emph{kernel} and runs in a privileged software mode that gives the kernel complete access to all hardware resources. The code running outside the kernel is often referred to as the \emph{user space} or the \emph{userland}, and has only restricted access to hardware. Because of this organization, all hardware-related activities necessary to run the operating system is performed by the kernel, and the user space programs only access hardware through \emph{system calls} in the kernel.\cite{modern-operating-systems}

\subsection{Device Drivers}
To more easily manage hardware devices with different characteristics, the kernel contains \emph{device drivers}. A device driver is a program that manages low-level hardware access to a particular device, providing a clean interface for the rest of the kernel programs.

\subsection{Kernel Modules}
It is sometimes necessary to extend the functionality of the kernel, for instance if new hardware becomes available. While this could be achieved by modifying and rebuilding the kernel, a much more attractive alternative is to use \emph{kernel modules}, small programs that are loaded at runtime and extend the kernel with the needed functionality. Device drivers can be added as kernel modules.

\subsection{Related Terminology}
In addition to the concepts described above, several other terms are used in the context of operating systems:
\begin{itemize}
\item \textsl{Boot Loader:} The bootloader is a small program that runs before the operating system starts, and makes the necessary preparations to start the kernel.
\item \textsl{Linux Root Filesystem:} In Linux, the root filesystem is the filesystem available at the top-level directory. It is denoted with a forward slash, "/".
\end{itemize}

