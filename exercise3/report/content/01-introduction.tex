\chapter{Introduction}
% Your report should start with an introduction chapter that motivates the
% subject in general and describes the problem you are trying to solve.

Embedded computing systems are being deployed on an increasingly larger scale, and it is expected that this trend will continue in the forseable future. It follows that there is a great market for developing software for embedded computing systems. In contrast to general-purpose computing, embedded computing often places strong constraints on energy consumption, performance, and cost. In order to implement these qualities in a system, software developers must make extensive use of low-level hardware functionality, and specific knowledge of the hardware components and technology used is a requirement. 

TDT4258 Energy Efficient Computer Systems is a university-level course at the Norwegian University of Science and Technology.
The course gives an introduction to low-level microcontroller programming with a strong focus on energy-efficiency. Through a series of lectures and three comprehensive programming exercises, the students get theoretical knowledge as well as practical experience.\cite{ntnu-web-tdt4258} The target platform used is the Silicon Labs EFM32GG-DK3750 development kit, and some of the tools used are the GNU Compiler Collection, the PTXdist build system from Pengutronix, and the energyAware Tools from Silicon Labs.\cite{compendium}

In the third exercise of the course the students make a computer game for the prototyping board. The game should run under the uClinux operating system, and all hardware must be accessed through device drivers. The game should also make use of a special prototype gamepad, and a gamepad driver must be written to access it. An important exercise goal is that the program should be energy-efficient. To improve energy consumption and to utilize the limited hardware resources well, the program must use appropriate algorithms and programming techniques, and may also need to use special hardware functions. To learn the required programming techniques, the students may have to study relevant theory on their own.\cite{compendium}

