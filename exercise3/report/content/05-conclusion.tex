\chapter{Conclusion}
In this exercise we have compiled an OS and implemented the necessary drivers to interact with the board via a gamepad. 

On these foundations we have developed a fairly complex game running in real time with a lot of calculations, taxing the recourses of the board heavily. Even with the relatively large demands of our game, and the added overhead of running an operating system, we have shown that hitting 30 frames per second with a fairly complex vector based game is possible.

\section{Evaluation of the Assignment}
% You can include comments about the assignment itself here. While this part is
% not obligatory and not graded, it is valuable feedback to the course staff
% that can be used to improve the exercises in the future.

\subsection{Compendium Feedback}
This section contains feedback and suggested improvements for the compendium.

\subsubsection{Mention \texttt{ptxdist clean} in section 5.3.3}
As mentioned in the compendium in section \textbf{5.3.3. Building and Flashing}, a single package can be flashed with the following four commands:
\begin{enumerate}
  \item \texttt{ptxdist compile <packagename>, e.g game or driver-gamepad}
  \item \texttt{ptxdist targetinstall <packagename>}
  \item \texttt{ptxdist image root.romfs}
  \item \texttt{ptxdist test flash-rootfs}
\end{enumerate}
However, in order for the modified sources to be compiled, we also needed run one of the commands \texttt{ptxdist clean game} and \texttt{ptxdist drop game compile}. Maybe it would be useful to mention this in this section.

