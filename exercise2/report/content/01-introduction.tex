\chapter{Introduction}
% Your report should start with an introduction chapter that motivates the
% subject in general and describes the problem you are trying to solve.

% TODO: Motivate the subject in general

% The exercise itself and requirements
In this exercise we will write a program for the EFM32GG-DK3750 that generates sound effects by utilising the microcontroller's DAC (Digital to Analog Converter). A prototype gamepad will be connected that allows the user to control the sound by pressing different buttons. We need to make at least three different sound effects and a start up melody that can be used in a game. Extra points will be given for an energy efficient solution. The program is written in C, and should run directly on the board with no operating system involved. As an aid to setting up the hardware, some skeleton code was provided with function stubs and several macros for useful register addresses.

% Practical approach and learning outcome
We will use the GNU Compiler Collection to compile and link source files. As there are several useful C libraries we may want to link, we need to pay extra attention to the linking process. We also need to program several board peripherals, including the DAC, the PRS (Peripheral Reflex System), an on-board timer, and the GPIO (General-Purpose Input/Output) pins. The purpose of the exercise is to learn about C programming, GPIO control in C, interrupt handling in C, and to generate sound effects using the microcontroller's DAC and timers.

